\begin{abstract}

Historically, operating systems used for critical system applications in aerospace, such as commercial airplanes and satellites, and outer space, have been restricted to real-time commercial OSes.

Although commercial OS used in this domain deliver both commercial support, real-time performance, and a high level of assurance, they also carry with them a number of limitations. Using a commercial OS normally requires a mature organization that has purchasing/contracting agility for rapid development and the ability to incorporate the OS artifacts for the level of certification required. Generally, a real-time commercial OS is prohibitively expensive for smaller organizations, educational institutions, and teams working on initial prototypes. Not having full source and artifact access limits teams' ability to fix and extend a system, innovate, and mature technology.

This paper presents a collaborative survey, conducted under the umbrella of the ELISA foundation (Enabling Linux in Safety Critical Applications), on state-of-the-art use of Linux in aircraft, spacecraft, or vehicles operating in similar constrained environments. The paper outlines the current state of the eco-system used in industry, and the relationships between systems, service history, and regulatory standards.

While the paper's focus is on Linux in particular, other operating systems that are Unix-like or within a partition hosted by some kind of hypervisor are investigated for comparison as well.

\end{abstract}
